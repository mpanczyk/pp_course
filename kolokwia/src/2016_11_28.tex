\documentclass[extrafontsizes,12pt]{article}
\usepackage{polski}
\usepackage[utf8]{inputenc}
\usepackage{amssymb}
\usepackage{amsfonts}
\usepackage{stmaryrd}
\usepackage{amsmath}
\usepackage{fancyvrb}
\usepackage{graphicx}
\usepackage{psfrag}
\usepackage{wrapfig}
\usepackage{relsize}

\usepackage[a4paper,left=2cm,right=2cm,top=1.5cm,bottom=2cm]{geometry}
\sloppy

\title{Podstawy Programowania --- kolokwium I}
\date{28 listopada 2016}

\begin{document}

\maketitle
\DefineShortVerb{\|}
\thispagestyle{empty}

\begin{enumerate}
\itemsep1em


\item \textbf{(10 pkt)}
Napisz program, który wczyta od~użytkownika jego wzrost~$h$ (w~$m$) oraz~masę~$m$ (w~$kg$).
Program ma obliczyć indeks masy ciała ($BMI$) według wzoru: $BMI = m / h ^ 2$.
Program ma wypisać wynik na~ekranie,
a~następnie wyświetlić informację o~wadze ciała użytkownika według reguły:

Dla $BMI \le 18.5$ program ma wyświetlić komunikat ,,niedowaga'',\\
Dla $18.5 < BMI \le 25$ program ma wyświetlić komunikat ,,prawidłowa waga'',\\
Dla $25 < BMI \le 30$ program ma wyświetlić komunikat ,,lekka nadwaga'',\\
Dla $30 < BMI$ program ma wyświetlić komunikat ,,nadwaga''.


\item \textbf{(10 pkt)}
Napisz funkcję, która przyjmuje jako parametry współrzędne trzech punktów:
$A = (x_A, y_A)$, $B = (x_B, y_B)$, $C = (x_C, y_C)$.
Punkty te tworzą trójkąt.
Funkcja ma zwrócić wartość proporcji najkrótszego boku do~najdłuższego boku trójkąta.


\item \textbf{(10 pkt)}
Napisz rekurencyjną funkcję obliczającą $n$-ty wyraz ciągu według podanego niżej wzoru.
\begin{equation*}
a(n) =
	\begin{cases}
		n  & \mbox{ dla } n < 6,\\
		\mathlarger{\sum_{i=1}^{3}} a(n-2i) & \mbox{ dla } n \ge{} 6.
	\end{cases}
\end{equation*}


\item \textbf{(10 pkt)}
Napisz program, który wczytuje liczbę dodatnią $n$,
a następnie wczytuje $n$~liczb,
których możliwe wartości to 0 oraz 1.
Potraktuj otrzymany ciąg zer i~jedynek jak liczbę zapisaną
w~systemie dwójkowym podaną od~najstarszego bitu.
Wyświetl tę liczbę w~systemie dziesiętnym.

Przykład:
\begin{center}
  \begin{tabular}{ l l }
  wejście & wyjście \\
  \hline
  |5| & |22| \\
  |1 0 1 1 0|
  \end{tabular}
\end{center}

\end{enumerate}

\vfill

\textbf{Uwagi}

\begin{itemize}
\item W~każdym zadaniu (także w~tych, w~których trzeba napisać tylko funkcję) należy dopisać pliki nagłówkowe, z~których korzystamy.
\item Rozwiązanie każdego zadania może zawierać dowolną liczbę funkcji pomocniczych.
\item Zakładamy, że dane wejściowe spełniają określone w treści zadania warunki, więc nie~trzeba sprawdzać ich poprawności.
\item Prace nieczytelne nie będą sprawdzane.
\item Każde zadanie należy rozwiązać na~osobnej, podpisanej kartce. Wszystkie kartki (nawet puste) należy oddać.
\end{itemize}
\end{document}
