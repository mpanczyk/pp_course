\documentclass[extrafontsizes,10pt]{article}
\usepackage{polski}
\usepackage[utf8]{inputenc}
\usepackage{amssymb}
\usepackage{amsfonts}
\usepackage{stmaryrd}
\usepackage{amsmath}
\usepackage{fancyvrb}
\usepackage{psfrag}
\usepackage{wrapfig}

\usepackage[a4paper,left=2cm,right=2cm,top=1.0cm,bottom=2cm]{geometry}
\sloppy

\title{Podstawy Programowania --- Kolokwium II}
\date{23 stycznia 2017}

\begin{document}

\maketitle
\DefineShortVerb{\|}
\thispagestyle{empty}
\begin{enumerate}
\itemsep1em

\item \textbf{(5 pkt)}
Napisz funkcję, która przyjmuje jako parametry
2 wskaźniki na~liczby rzeczywiste
i~oblicza podłogę z~iloczynu wartości,
na~które wskazują te wskaźniki.
Funkcja ma zwrócić wskaźnik
na~zmienną całkowitą przechowującą
tak otrzymaną wartość.

\item \textbf{(10 pkt)}
Napisz funkcję, która dla~podanego
parametrze łańcucha zwróci łańcuch,
w~którym litery z~alfabetu łacińskiego
zostaną zamienione:
duża na~małą zawsze,
a~mała na~dużą tylko wtedy,
gdy była na~początku wyrazu,
np. dla~łańcucha ,,Dark Side of the Moon''
zwrócony zostanie łańcuch
,,dark side Of The moon''.
Załóż, że~pierwsza litera wyrazu
znajduje się albo po~spacji,
albo na~początku łańcucha.
W~rozwiązaniu nie należy używać
funkcji z~biblioteki |string.h|.

\item \textbf{(10 pkt)}
Napisz funkcję, która jako parametry
otrzymuje dynamiczną 2-wymiarową
tablicę liczb zmiennoprzecinkowych
|tab| oraz~jej wymiary: |n| i~|m|.
Funkcja ma za~zadanie obliczyć i~zwrócić
sumę elementów znajdujących się
na~brzegach tablicy
(w~pierwszym i~ostatnim wierszu
oraz~w~pierwszej i~ostatniej kolumnie).
Dla~przykładowej tablicy:
$$tab =
\left[
  \begin{tabular}{cccc}
  1.0 2.0 3.0 4.0 5.0 \\
6.0 0.0 4.0 3.0 1.0 \\
7.0 3.0 4.0 6.0 2.0 \\
8.0 9.0 3.0 4.0 3.0 \\
  \end{tabular}
\right]
$$
\noindent prawidłową odpowiedzią będzie 58.0.


\item \textbf{(10 pkt)}
Napisz funkcję, która pobiera
jako parametry dynamiczną 2-wymiarową
tablicę liczb całkowitych |tab1|
oraz~jej wymiary: |n| i~|m|.
Funkcja ma dynamicznie przydzielić
pamięć na~tablicę |tab2|,
która ma mieć identyczne wymiary,
jak tablica |tab1|.
Następnie funkcja ma przepisać dane z~tablicy
|tab1| do~|tab2| w~taki sposób,
iż jeśli w~|tab1| pod indeksem
($i$,~$j$) znajduje się liczba nieparzysta,
wówczas w~tablicy |tab2| pod~indeksem
($i$, $j$) ma znaleźć się liczba
o~1 większa, natomiast jeśli jest parzysta,
ma zostać pomnożona przez~3
i~zapisana do~tablicy |tab2|.
Funkcja ma zwrócić wskaźnik
do~utworzonej tablicy |tab2|.

\item \textbf{(15 pkt)}
Napisz funkcję, która przyjmuje
jako parametry dynamiczną jednowymiarową
tablicę liczb całkowitych |tab|
oraz~jej rozmiar: |n|.
Funkcja ma zwrócić:
\begin{itemize}
\item
  1, jeśli każdy element tej tablicy
  o~wartości $k$ występuje w~niej
  dokładnie $k$ razy,
\item 0 w~przeciwnym przypadku.
\end{itemize}
\noindent Przetestuj napisaną funkcję w programie głównym.

\end{enumerate}

\vfill

\textbf{Uwagi}

\begin{itemize}
 \item W~każdym zadaniu (także w~tych,
       w~których trzeba napisać tylko funkcję)
       należy dopisać pliki nagłówkowe,
       z~których korzystamy.
 \item Rozwiązanie każdego zadania może zawierać
       dowolną liczbę funkcji pomocniczych.
 \item Zakładamy, że dane wejściowe spełniają
       określone w~treści zadania warunki,
       więc nie~trzeba sprawdzać ich poprawności.
 \item Prace nieczytelne nie będą sprawdzane.
 \item Każde zadanie należy rozwiązać na~osobnej,
       podpisanej kartce.
       Wszystkie kartki (nawet puste) należy oddać.
\end{itemize}

\end{document}
