\documentclass[extrafontsizes,10pt]{article}
\usepackage{polski}
\usepackage[utf8]{inputenc}
\usepackage{amssymb}
\usepackage{amsfonts}
\usepackage{stmaryrd}
\usepackage{amsmath}
\usepackage{fancyvrb} 
\usepackage{graphicx}
\usepackage{psfrag}
\usepackage{wrapfig}

\usepackage{titling}
\setlength{\droptitle}{-2em}
\posttitle{\par\end{center}\vspace{-1em}}

\usepackage[a4paper,left=2cm,right=2cm,top=1.0cm,bottom=2cm]{geometry}
\sloppy

\title{Podstawy Programowania --- kolokwium II}
\date{28 stycznia 2015}

\begin{document}

\maketitle
\DefineShortVerb{\|}
\thispagestyle{empty}
\begin{enumerate}
\itemsep1em


\item \textbf{(5 pkt)}
Napisz funkcję, która otrzymuje jako argumenty dwie liczby całkowite $n$ oraz $m$,
a następnie rezerwuje pamięć dla pojedynczej zmiennej typu całkowitego
oraz przypisuje jej sumę wartości otrzymanych argumentów i zwraca wskaźnik do niej.


\item \textbf{(7 pkt)}
Napisz makro $MIN6(a,b,c,d,e,f)$, korzystające z makra MIN2(x,y)
w celu zwrócenia najmniejszej z 6 liczb.
(Oznacza to, że makro $MIN6()$ może w swojej
definicji wywoływać makro $MIN2()$ dowolną ilość razy.)


\item \textbf{(11 pkt)}
Napisz funkcję, która przyjmuje wskaźniki do 3 tablic
liczb całkowitych $tab1$, $tab2$, $tab3$
oraz jeden argument będący długością tablic
(wszystkie tablice mają tę samą długość).
Funkcja ma za zadanie zaalokować nową tablicę $tab4$
tej samej długości, do której pod $i$-tym indeksem
ma wpisać liczbę kryjącą się pod tym samym indeksem
z $tab2$ lub $tab3$ w zależności od wartości znajdującej się
pod $i$-tym indeksem w tablicy $tab1$.
Jeśli w $tab1$ wartość jest większa lub równa $0$,
to ma wybrać wartość z $tab2$, w przeciwnym razie
z $tab3$. \underline{Przykład:} \\

\textbf{WE:} $tab4$: \{-1, 0, 3, -4, 10\}, $tab2$: \{1, 2, 3, 4, 5\}, $tab3$: \{10, 11, 12, 13, 14\}\\
\textbf{WY:} $tab4$: \{10, 2, 3, 13, 5\}


\item \textbf{(12 pkt)}
W celu zaoszczędzenia ilości znaków w krótkich
wiadomościach tekstowych (SMS) nie pisze się spacji,
a każdy wyraz rozpoczyna się z dużej litery.
Napisz funkcję, która jako argument otrzymuje wskaźniki
na dwie zaalokowane tablice: $napis$ (tekst oryginalny)
oraz $sms$ (wynikowy napis, który ma być zmodyfikowany
zgodnie z powyższym trendem).
Zakładamy, że między wyrazami są tylko pojedyncze spacje,
a po każdej spacji zaczyna się wyraz.
\underline{Podpowiedź:} kody ASCII wynoszą odpowiednio
(a-z: 97-122, A-Z: 65-90, spacja-32).
\underline{Przykład:} \\

\textbf{WE:} $napis$: ''It is not a bug, it is a feature'' \\
\textbf{WY:} $sms$: ''ItIsNotABug,ItIsAFeature'' \\

\item \textbf{(15 pkt)}
Zdefiniuj strukturę o nazwie $STUDENT$, która zawiera 2 pola:
dynamicznie zaalokowaną jednowymiarową tablicę elementów typu
int przechowującą oceny danego studenta - $oceny$
(tablica jest reprezentowana przez wskaźnik)
oraz liczbę całkowitą $ilosc$, wskazującą ilość ocen danego studenta.
Napisz funkcję, która dostaje 2 argumenty:
tablicę struktur $STUDENT$ o nazwie $tab$
oraz liczbę całkowitą $n$ wskazującą na ilość studentów.
Tablica $tab$ przechowuje dane o ocenach $n$-kandydatów na studia.
Indeksy elementów tablicy $tab$ są jednocześnie numerami kandydatów.
Na studia przyjmowane są osoby, których średnia ocen jest
większa lub równa $4.0$.
Funkcja powinna wyświetlić listę osób przyjętych na studia
(numer kandydata oraz średnią jego ocen).
\underline{Przykład:} \\

\textbf{WE:} $n$: 3, $tab[0]$: { {ilosc: 2, oceny: [3, 4]}, $tab[1]$: {ilosc: 4, oceny: [3, 4, 5, 4]},  $tab[2]$:
{ilosc: 1, oceny: [5]}},  \\
\textbf{WY:} 1 - 4.0, 2 - 5.0  


\end{enumerate}

\vfill

\textbf{Uwagi}

\begin{itemize}
 \item W~każdym zadaniu (także w~tych, w~których trzeba napisać tylko funkcję) należy dopisać pliki nagłówkowe, z~których korzystamy.
 \item Rozwiązanie każdego zadania może zawierać dowolną liczbę funkcji pomocniczych.
 \item Zakładamy, że dane wejściowe spełniają określone w treści zadania warunki, więc nie~trzeba sprawdzać ich poprawności.
 \item Prace nieczytelne nie będą sprawdzane.
 \item Każde zadanie należy rozwiązać na~osobnej, podpisanej kartce. Wszystkie kartki (nawet puste) należy oddać.
\end{itemize}

\end{document}
