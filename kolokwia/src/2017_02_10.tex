\documentclass[extrafontsizes,10pt]{article}
\usepackage{polski}
\usepackage[utf8]{inputenc}
\usepackage{amssymb}
\usepackage{amsfonts}
\usepackage{stmaryrd}
\usepackage{amsmath}
\usepackage{fancyvrb}
\usepackage{graphicx}
\usepackage{psfrag}
\usepackage{wrapfig}

\usepackage{titling}
\setlength{\droptitle}{-2em}
\posttitle{\par\end{center}\vspace{-1em}}

\usepackage[a4paper,left=2cm,right=2cm,top=1.0cm,bottom=2cm]{geometry}
\sloppy

\title{Podstawy Programowania --- Kolokwium III}
\date{10 lutego 2017}

\begin{document}

\maketitle
\DefineShortVerb{\|}
\thispagestyle{empty}
\begin{enumerate}
\itemsep1em

\item \textbf{(15 pkt)}
Zdefiniuj strukturę |Deltoid|, która przechowuje długości przekątnych deltoidu.
Napisz funkcję, która przyjmuje jako parametr zmienną typu |Deltoid|
i~zwraca pole tak przekazanej figury.
Pole deltoidu to połowa iloczynu długości przekątnych.

\item \textbf{(15 pkt)}
Napisz funkcję, która przyjmuje jako parametry dynamiczną tablicę
dwuwymiarową liczb rzeczywistych oraz~dwie liczby całkowite |n| i~|m|,
które są wymiarami tej tablicy.
Funkcja ma zwrócić wskaźnik na~nowoutworzoną dynamiczną tablicę,
która przechowuje macierz transponowaną do~macierzy
przechowywanej w~przekazanej parametrem tablicy.

\item \textbf{(20 pkt)}
Napisz funkcję, która dla~podanej jako parametr tablicy liczb typu |int|
oraz~jej rozmiaru~|n|, zwróci wskaźnik na~kopię wartości największej
nieparzystej i~podzielnej przez~3 wartości z~tej tablicy.
Jeśli takiej wartości nie ma, funkcja powinna zwrócić |NULL|.
Uwaga! W~celu określenia, czy wartość jest nieparzysta i~podzielna przez~3,
należy napisać makro |NPARZ3(x)|, przyjmujące wartość~1, jeśli liczba spełnia
warunki i~0, jeśli ich nie~spełnia.

\noindent
Przykład:
Dla~tablicy \{0, 3, 2, 15, 6, 1, 7, 9, 20, 22, 25\} funkcja powinna zwrócić wskaźnik
na~kopię wartości 15.

\item \textbf{(25 pkt)}
Szeherezada każdego dnia spisywała opowieści, którymi się dzieliła
poprzedniej nocy z~sułtanem.
Gdy~już minęły wszystkie noce, a~sułtan darował jej życie,
postanowiła poukładać je w~odpowiednim porządku.
Niestety jest ich tak wiele, że~warto byłoby jej pomóc.
Napisz funkcję, która dla~podanej jako parametr
tablicy napisów i~rozmiaru tej tablicy w~drugim parametrze,
zwróci tablicę napisów, w~których ostatnie zdanie,
poprzedzone spacją, będzie kopią pierwszego zdania z~następnego łańcucha.
Zdanie to, wraz z~następującą po~nim spacją, należy usunąć z~następnego łańcucha.
Usunięcie pierwszego zdania nie dotyczy pierwszego napisu w~tablicy,
zaś dołączenie ostatniego nie dotyczy ostatniego elementu tablicy.
Uznaj, że~zdanie to sekwencja znaków zakończona kropką.

\noindent
Przykład:
Dla~tablicy: 
\{''Ala ma kota. Kot ma Asa. As ma w głowie pstro.'',
''Ala lubi psa. Beata lubi kota. Kot lubi Alę.'',
''Beata lubi Alę. Czesława to babcia Beaty.''\}
zostanie zwrócona tablica
\{''Ala ma kota. Kot ma Asa. As ma w~głowie pstro. Ala lubi psa.'',
''Beata lubi kota. Kot lubi Alę. Beata lubi Alę.'',
''Czesława to babcia Beaty.''\}


\item \textbf{(25 pkt)}
Napisz funkcję, która w parametrach dostaje tablicę struktur |Student|
o~nazwie |tab| oraz~liczbę jej elementów~|n|.
Struktura składa się z~trzech pól: dynamicznej tablicy liczb zmiennoprzecinkowych
o~nazwie |oceny|, liczby ocen~|m| oraz~napisu, będącego nazwiskiem studenta.
Funkcja ma zwrócić nazwisko studenta z~najwyższą średnią ocen.

\end{enumerate}

\vfill

\textbf{Uwagi}

\begin{itemize}
 \item  W~każdym zadaniu (także w~tych, w~których trzeba napisać tylko funkcję)
        należy dopisać pliki nagłówkowe, z~których korzystamy.
 \item  Rozwiązanie każdego zadania może zawierać dowolną liczbę funkcji pomocniczych.
 \item  Zakładamy, że dane wejściowe spełniają określone w treści zadania warunki,
        więc nie~trzeba sprawdzać ich poprawności.
 \item  Prace nieczytelne nie będą sprawdzane.
 \item  Każde zadanie należy rozwiązać na~osobnej, podpisanej kartce.
        Wszystkie kartki (nawet puste) należy oddać.
\end{itemize}

\end{document}

