\documentclass[11pt]{article}
\usepackage{polski}
\usepackage[utf8]{inputenc}
\usepackage{amssymb}
\usepackage{amsfonts}
\usepackage{stmaryrd}
\usepackage{amsmath}
\usepackage{fancyvrb} 
\usepackage{graphicx}
\usepackage{psfrag}
\usepackage{wrapfig}

\usepackage[a4paper,left=2cm,right=2cm,top=1cm,bottom=1.5cm]{geometry}
\sloppy

\title{Podstawy Programowania --- kolokwium I}
\date{6 grudnia 2013}

\begin{document}

\maketitle
\DefineShortVerb{\|}
\thispagestyle{empty}
\begin{enumerate}


\item \textbf{(7 pkt)}
Napisz program, który pobierze od użytkownika 6 liczb zmiennoprzecinkowych będących współrzędnymi ($x$, $y$) trzech punktów na płaszczyźnie.
Jeżeli leżą one na jednej prostej, program powinien wyświetlić współczynniki $a$, $b$ równania tej prostej w~postaci $y=ax+b$.
W~przeciwnym razie program powinien wyświetlić komunikat ,,niewspółliniowe''. Żadne dwa punkty nie mają równej współrzędnej~$x$.

 \item \textbf{(7 pkt)}
Napisz rekurencyjną funkcję, która dla podanych $k$ i~$n$ oblicza $k$-ty wyraz ciągu danego wzorem:
$$
\begin{cases}
  c_{0} & = n \\
  c_{k} & = \frac{c_{k-1}}{2} \mbox{ dla } c_{k-1} \mbox{ parzystego} \\
  c_{k} & = 3 \cdot c_{k-1}+1 \mbox{ dla } c_{k-1} \mbox{ nieparzystego.}
\end{cases}
$$
Napisz program, który wczytuje od~użytkownika liczbę $n$ i~$k$, a~następnie oblicza $k$-ty wyraz ciągu $(c_k)$ i~go wyświetla.
Nie można używać zmiennych globalnych.

 \item \textbf{(8 pkt)}
Deweloper buduje osiedle mieszkaniowe.
Ma do~swojej dyspozycji $n$ ekip budowlanych. Z~Urzędu Miasta przychodzi nagła decyzja o~zakazie budowania obiektów wyższych niż $p$ pięter.
Niestety, jedne budynki mają już o~wiele więcej pięter niż~$p$, inne są niższe niż~$p$.
%Na~jednym placu budowy mieści się tylko jedna ekipa.
W~ciągu dnia jedna ekipa może postawić lub zdemontować jedno piętro.
Napisz funkcję, która przyjmuje tablicę całkowitoliczbowych wysokości budynków i~ich liczbę $n$ (długość tablicy), która jest jednocześnie liczbą ekip oraz liczbę $p$.
Funkcja ma zwrócić liczbę dni, które upłyną zanim wszystkie budynki będą miały określoną przez~Urząd wysokość~$p$.


 \item \textbf{(9 pkt)}
Napisz funkcję, która przyjmuje jako parametry dwie tablice zmiennoprzecinkowe, ich wspólny rozmiar oraz liczby $\xi$ i~$\zeta$ ($\xi < \zeta$).
Funkcja ma skopiować z~pierwszej tablicy do~drugiej wszystkie wartości należące do~przedziału domkniętego $[\xi; \zeta]$ oraz zwrócić liczbę skopiowanych elementów.
Kopiowane liczby mają być umieszczone jedna po~drugiej na~początku drugiej tablicy.


 \item \textbf{(9 pkt)}
Napisz funkcję, która przyjmuje jako parametr tablicę liczb całkowitych dodatnich i~jej rozmiar.
Funkcja traktuje tablicę jako planszę do~gry składającą się z~pól, którymi są elementy tablicy indeksowane od~0.
Gra rozpoczyna się na~polu o~numerze~0.
Funkcja wykonuje ruch przesuwając się o~tyle pól w~prawo, ile zapisane jest w~aktualnym polu.
Funkcja powinna zwrócić liczbę ruchów, które wykonała aby opuścić planszę.

\end{enumerate}

\textbf{Uwagi}

\begin{itemize}
 \item W każdym zadaniu (także w tych, w których trzeba napisać tylko funkcję) należy dopisać nagłówki bibliotek, z których korzystamy.
 \item Rozwiązanie każdego zadania może zawierać dowolną liczbę funkcji pomocniczych.
 \item Zakładamy, że dane wejściowe spełniają określone w treści zadania warunki, więc nie~trzeba sprawdzać ich poprawności.
 \item Prace nieczytelne nie będą sprawdzane.
 \item Każde zadanie należy rozwiązać na osobnej, podpisanej kartce. Wszystkie kartki (nawet puste) należy oddać.
\end{itemize}

\end{document}
