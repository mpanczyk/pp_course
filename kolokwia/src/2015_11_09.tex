\documentclass[extrafontsizes,12pt]{article}
\usepackage{polski}
\usepackage[utf8]{inputenc}
\usepackage{amssymb}
\usepackage{amsfonts}
\usepackage{stmaryrd}
\usepackage{amsmath}
\usepackage{fancyvrb}
\usepackage{graphicx}
\usepackage{psfrag}
\usepackage{wrapfig}

\usepackage[a4paper,left=2cm,right=2cm,top=1.5cm,bottom=2cm]{geometry}
\sloppy

\title{Podstawy Programowania --- kolokwium I}
\date{9 listopada 2015}

\begin{document}

\maketitle
\DefineShortVerb{\|}
\thispagestyle{empty}

\begin{enumerate}
\itemsep1em


\item \textbf{(10 pkt)}
Napisz program, który wczyta od~użytkownika 2~liczby całkowite: $x$ oraz~$y$,
będące współrzędnymi punktu~$P$ na~płaszczyźnie.
Następnie obliczy i~wyświetli wartość cosinusa kąta,
którego pierwsze ramię zawiera dodatnią półoś~$OX$,
a~drugie przechodzi przez punkt $P=(x, y)$.
Wierzchołkiem kąta jest początek układu współrzędnych.

\item \textbf{(10 pkt)}
Napisz funkcję, która jako argument przyjmuje liczbę całkowitą~$x$
i~zwraca \textbf{liczbę jej cyfr} podzielnych przez~3.

\item \textbf{(10 pkt)}
Zaimplementuj następującą funkcję rekurencyjną:
\begin{equation*}
 N_b = \begin{cases}
    0 & \mbox{ dla } b = 0, \\
    6 & \mbox{ dla } b = 1, \\
    3 + N_{b-1} & \mbox{ dla } b > 1.
       \end{cases}
\end{equation*}

\item \textbf{(10 pkt)}
Napisz funkcję, która przyjmuje w~argumencie liczbę naturalną~$n>1$
i~zwraca
\begin{itemize}
  \item 1 jeśli podana liczba jest liczbą pierwszą,
  \item 0 jeśli podana liczba jest liczbą złożoną.
\end{itemize}

\end{enumerate}

\vfill

\textbf{Uwagi}

\begin{itemize}
\item W~każdym zadaniu (także w~tych, w~których trzeba napisać tylko funkcję) należy dopisać pliki nagłówkowe, z~których korzystamy.
\item Rozwiązanie każdego zadania może zawierać dowolną liczbę funkcji pomocniczych.
\item Zakładamy, że dane wejściowe spełniają określone w treści zadania warunki, więc nie~trzeba sprawdzać ich poprawności.
\item Prace nieczytelne nie będą sprawdzane.
\item Każde zadanie należy rozwiązać na~osobnej, podpisanej kartce. Wszystkie kartki (nawet puste) należy oddać.
\end{itemize}
\end{document}
